\documentclass[conference]{IEEEtran}
\IEEEoverridecommandlockouts
% The preceding line is only needed to identify funding in the first footnote. If that is unneeded, please comment it out.
\usepackage{cite}
\usepackage{amsmath,amssymb,amsfonts}
\usepackage{algorithmic}
\usepackage{graphicx}
\usepackage{textcomp}
\usepackage{xcolor}
\def\BibTeX{{\rm B\kern-.05em{\sc i\kern-.025em b}\kern-.08em
    T\kern-.1667em\lower.7ex\hbox{E}\kern-.125emX}}
\begin{document}

\title{The Pythonian Hut Program\\
{\footnotesize Professional Reflection}
\thanks{Identify applicable funding agency here. If none, delete this.}
}

\author{\IEEEauthorblockN{1\textsuperscript{st} Nguyen Quang Huy Pham}
\IEEEauthorblockA{\textit{Royal Melbourne Insitue of Technology} \\
\textit{Sid:4181834}\\
Melbourne, Autralia \\
s4181834@student.rmit.edu.au}

}

\maketitle

\section{Introduction}


\section{Program Design}
\subsection{Part 1 - Basic Booking Flow}
\subsubsection{Prompts and Variable Storage}
The program prompt user for guest name, number of guests, apartment ID, check in and check out date, length of stay and booking date. All of the input is validated such as the guest input can not be empty, number of guest must be integer. All of the Variables are stored and passes to the booking process. This has meet the rubic with clear message and minor limitation no consistency check between dates and length of stay 
\subsubsection{Data Types and Initialization}
The data type and inilization is clearly defined in the code base. Guest and point are stored in dictionary \textbf{picture}. Apartment and rates stored in dictionary \textbf{picture}. The program first initialize with the required data as assignment details (Alyssa (20), Luigi (32)) and three apartments with nightly rates. The program have implemented a good use of data type, only with the minor issue when using StorageManager, new guest are not auto-added
\subsubsection{Cost and Rewards Caculation}
The cost and reward caculation follow stricly based on the assignment details, cost = nightly rates x length of stay. Rewards = total cost rounded half-up to the nearest integer.
\subsubsection{Receipt Formatting}
The receipt prints all required details including guest, number of guest, apartment, rate, check in and check put date, nights, booking date, total cost and reward points. The format is clear and consistent with the assignment specification with AUD currency shown to 2 decimals
\subsubsection{Beyond Requirement Features}
The logging to file and stdout, storage manager with add/update/delete operations. Also there is a interactive booking loop 
\subsection{Functional Requirements}

\subsection{Data Structures}

\subsection{Input Validation}

\section{Flowchart}

\section{Implemetation Highlight}

\section{Testing \& Result}

\section{Reflection}



\section*{Acknowledgment}

The preferred spelling of the word ``acknowledgment'' in America is without 
an ``e'' after the ``g''. Avoid the stilted expression ``one of us (R. B. 
G.) thanks $\ldots$''. Instead, try ``R. B. G. thanks$\ldots$''. Put sponsor 
acknowledgments in the unnumbered footnote on the first page.

\section*{References}

Please number citations consecutively within brackets \cite{b1}. The 
sentence punctuation follows the bracket \cite{b2}. Refer simply to the reference 
number, as in \cite{b3}---do not use ``Ref. \cite{b3}'' or ``reference \cite{b3}'' except at 
the beginning of a sentence: ``Reference \cite{b3} was the first $\ldots$''

Number footnotes separately in superscripts. Place the actual footnote at 
the bottom of the column in which it was cited. Do not put footnotes in the 
abstract or reference list. Use letters for table footnotes.

Unless there are six authors or more give all authors' names; do not use 
``et al.''. Papers that have not been published, even if they have been 
submitted for publication, should be cited as ``unpublished'' \cite{b4}. Papers 
that have been accepted for publication should be cited as ``in press'' \cite{b5}. 
Capitalize only the first word in a paper title, except for proper nouns and 
element symbols.

For papers published in translation journals, please give the English 
citation first, followed by the original foreign-language citation \cite{b6}.

\begin{thebibliography}{00}
\bibitem{b1} G. Eason, B. Noble, and I. N. Sneddon, ``On certain integrals of Lipschitz-Hankel type involving products of Bessel functions,'' Phil. Trans. Roy. Soc. London, vol. A247, pp. 529--551, April 1955.
\bibitem{b2} J. Clerk Maxwell, A Treatise on Electricity and Magnetism, 3rd ed., vol. 2. Oxford: Clarendon, 1892, pp.68--73.
\bibitem{b3} I. S. Jacobs and C. P. Bean, ``Fine particles, thin films and exchange anisotropy,'' in Magnetism, vol. III, G. T. Rado and H. Suhl, Eds. New York: Academic, 1963, pp. 271--350.
\bibitem{b4} K. Elissa, ``Title of paper if known,'' unpublished.
\bibitem{b5} R. Nicole, ``Title of paper with only first word capitalized,'' J. Name Stand. Abbrev., in press.
\bibitem{b6} Y. Yorozu, M. Hirano, K. Oka, and Y. Tagawa, ``Electron spectroscopy studies on magneto-optical media and plastic substrate interface,'' IEEE Transl. J. Magn. Japan, vol. 2, pp. 740--741, August 1987 [Digests 9th Annual Conf. Magnetics Japan, p. 301, 1982].
\bibitem{b7} M. Young, The Technical Writer's Handbook. Mill Valley, CA: University Science, 1989.
\end{thebibliography}
\vspace{12pt}
\color{red}
IEEE conference templates contain guidance text for composing and formatting conference papers. Please ensure that all template text is removed from your conference paper prior to submission to the conference. Failure to remove the template text from your paper may result in your paper not being published.

\end{document}
