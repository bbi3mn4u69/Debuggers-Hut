\documentclass[conference]{IEEEtran}
\IEEEoverridecommandlockouts
% The preceding line is only needed to identify funding in the first footnote. If that is unneeded, please comment it out.
\usepackage{cite}
\usepackage{amsmath,amssymb,amsfonts}
\usepackage{algorithmic}
\usepackage{graphicx}
\usepackage{textcomp}
\usepackage{xcolor}
\def\BibTeX{{\rm B\kern-.05em{\sc i\kern-.025em b}\kern-.08em
    T\kern-.1667em\lower.7ex\hbox{E}\kern-.125emX}}
\begin{document}

\title{The Pythonian Hut Program\\
{\footnotesize Professional Reflection}
}

\author{\IEEEauthorblockN{1\textsuperscript{st} Nguyen Quang Huy Pham}
\IEEEauthorblockA{\textit{Royal Melbourne Insitue of Technology} \\
\textit{Sid:4181834}\\
Melbourne, Autralia \\
s4181834@student.rmit.edu.au}
}

\maketitle

\section{Introduction}


\section{Program Design}
\subsection{Part 1 - Basic Booking Flow}
\subsubsection{Prompts and Variable Storage}
The program have successfully implement all the required feature of user prompts for guest name, number of guest, apartment ID, checkin date, checkout date, length of stay, and booking date. Each entry is captured from the user keyboard input and stored in the appropriate variable type. All the input have its own validation that ensure such common error like leaving the field blank or miss match input type will trigger a clear error messages to the user. This design guarantees robust data collection at the earliest stage of the booking process.
\subsubsection{Data Types and Initialization}
Guest details and apartment information are stored in a dictionaries for efficient lookups and update. Speciffically, guest name maps to thier own reward point and apartment id maps to its own nightly rates. The system initialize based on the assingment requirement with two existing guest Alyssa (20 points) and Luigi (32 points), along with three apartments (U12swan, U209duck, and U49goose) and their respective nightly rates.
\subsubsection{Cost and Rewards Caculation}
Align with the requirement of the assignment, the systemm calculates the total booking cost by multiplying the apartment's nightly rate by the length of stay. Reward points are awarded based on the total cost in AUD using a half-up rounding method.
\subsubsection{Receipt Formatting}
A biingling reciept is generated and displaed at the end of each booking process. The reciept inclued all of the required field  guest name, number of guests, apartment ID, apartment rate (AUD with two decimal places), check-in date, check-out date, length of stay, booking date, total cost, and reward points earned. The receipt formatting is clear and user-friendly, closely mirroring the layout provided in the assignment specification.
\subsection{Functional Requirements}

\subsection{Data Structures}

\subsection{Input Validation}

\section{Flowchart}

\section{Implemetation Highlight}

\section{Testing \& Result}

\section{Reflection}



\section*{Acknowledgment}

The preferred spelling of the word ``acknowledgment'' in America is without 
an ``e'' after the ``g''. Avoid the stilted expression ``one of us (R. B. 
G.) thanks $\ldots$''. Instead, try ``R. B. G. thanks$\ldots$''. Put sponsor 
acknowledgments in the unnumbered footnote on the first page.

\section*{References}

Please number citations consecutively within brackets \cite{b1}. The 
sentence punctuation follows the bracket \cite{b2}. Refer simply to the reference 
number, as in \cite{b3}---do not use ``Ref. \cite{b3}'' or ``reference \cite{b3}'' except at 
the beginning of a sentence: ``Reference \cite{b3} was the first $\ldots$''

Number footnotes separately in superscripts. Place the actual footnote at 
the bottom of the column in which it was cited. Do not put footnotes in the 
abstract or reference list. Use letters for table footnotes.

Unless there are six authors or more give all authors' names; do not use 
``et al.''. Papers that have not been published, even if they have been 
submitted for publication, should be cited as ``unpublished'' \cite{b4}. Papers 
that have been accepted for publication should be cited as ``in press'' \cite{b5}. 
Capitalize only the first word in a paper title, except for proper nouns and 
element symbols.

For papers published in translation journals, please give the English 
citation first, followed by the original foreign-language citation \cite{b6}.

\begin{thebibliography}{00}
\bibitem{b1} G. Eason, B. Noble, and I. N. Sneddon, ``On certain integrals of Lipschitz-Hankel type involving products of Bessel functions,'' Phil. Trans. Roy. Soc. London, vol. A247, pp. 529--551, April 1955.
\bibitem{b2} J. Clerk Maxwell, A Treatise on Electricity and Magnetism, 3rd ed., vol. 2. Oxford: Clarendon, 1892, pp.68--73.
\bibitem{b3} I. S. Jacobs and C. P. Bean, ``Fine particles, thin films and exchange anisotropy,'' in Magnetism, vol. III, G. T. Rado and H. Suhl, Eds. New York: Academic, 1963, pp. 271--350.
\bibitem{b4} K. Elissa, ``Title of paper if known,'' unpublished.
\bibitem{b5} R. Nicole, ``Title of paper with only first word capitalized,'' J. Name Stand. Abbrev., in press.
\bibitem{b6} Y. Yorozu, M. Hirano, K. Oka, and Y. Tagawa, ``Electron spectroscopy studies on magneto-optical media and plastic substrate interface,'' IEEE Transl. J. Magn. Japan, vol. 2, pp. 740--741, August 1987 [Digests 9th Annual Conf. Magnetics Japan, p. 301, 1982].
\bibitem{b7} M. Young, The Technical Writer's Handbook. Mill Valley, CA: University Science, 1989.
\end{thebibliography}
\vspace{12pt}
\color{red}
IEEE conference templates contain guidance text for composing and formatting conference papers. Please ensure that all template text is removed from your conference paper prior to submission to the conference. Failure to remove the template text from your paper may result in your paper not being published.

\end{document}
